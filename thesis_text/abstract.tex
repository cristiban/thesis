\begin{abstract}
Estimating neuronal stimulus-response functions remains a fundamental challenge in neuroscience. Traditionally, linear methods such as spectro-temporal receptive fields (STRFs) were used, but often fall short in capturing the complex nonlinear dynamics of neuronal responses to natural stimuli. This thesis addresses this limitation by employing deep spectro-temporal receptive fields (DSTRFs), a novel approach derived from the linearization of deep neural networks, to analyze calcium imaging data from the mouse auditory cortex.

Using calcium imaging with the jGCaMP8m indicator, neuronal activity in response to dynamic random chord (DRC) stimuli was recorded from the auditory cortex of mice. The preprocessed imaging data served as targets to train a convolutional neural network (CNN) that predicts neuronal activity from the auditory spectrograms. Subsequently, this trained CNN was linearized to extract interpretable DSTRFs for individual pixels over time, allowing for the visualization of spatial and temporal response kernels.

This work demonstrates that DSTRFs provide more insightful and interpretable estimates of neuronal response kernels compared to traditional linear methods. By generating a chronological "movie" of DSTRFs for each pixel, the analysis reveals deeper insights into the functional organization of auditory neurons and enhances our ability to predict neuronal responses to novel auditory stimuli. These findings contribute to a more comprehensive understanding of neural processing in the mouse auditory system and offer a powerful tool for future investigations into kernel response functions.
\end{abstract}
