\chapter{Background (optional)}\label{background}
The tonotopic organization of the auditory is well established in the literature, both in humans \parencite{saenzTonotopicMappingHuman2014}, and in other mammals \parencite{realeTonotopicOrganizationAuditory1980,bizleyFunctionalOrganizationFerret2005}. Linear methods such as STRF have been successfully used to estimate neuronal tuning in the form of frequency-time lag response kernels, based on the representation of acoustic signals. \parencite{aertsenSpectrotemporalReceptiveFields1980a}. Nevertheless, these linear methods fail to fully capture the intricacies of neuronal activity and lack robustness relative to the stimulus used \parencite{ahrensNonlinearitiesContextualInfluences2008}. Consequently, multiple nonlinear models have been proposed in order to improve accuracy \parencite{meyerModelsNeuronalStimulusResponse2017}. A more recently introduced such model is DSTRF, which is obtained by linearizing a deep neural network \parencite{keshishianEstimatingInterpretingNonlinear2020}. This method has been shown to be more accurate in fitting to nonlinear data, with the differences being more striking for increasingly nonlinear data. Moreover, the interpretability of this method allows for intuitive visualization of the tuning function, quite similar to STRF \parencite{keshishianEstimatingInterpretingNonlinear2020}.

In this paper, I will be applying this novel method to calcium imaging data, in the hope that new insights will emerge from the analysis.